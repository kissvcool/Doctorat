\documentclass[12pt,a4paper]{report}
  \usepackage[utf8]{inputenc}
  \usepackage[francais]{babel}
  \usepackage[T1]{fontenc}
  \usepackage{amsmath}
  \usepackage{cancel} % Barrer des éléments mathématiques
  \usepackage{pgfplots}  % Tikz
  \usepackage[linkcolor=black,colorlinks=true]{hyperref}
  
\begin{document}


\renewcommand{\contentsname}{Sommaire}
\tableofcontents

\section{Système de départ, TDG en dynamique}

À l'instant $m$ :
\begin{equation}
\begin{array}{c}
		\begin{bmatrix}   
		   		\mathbf{K}
			&
		   		0
		   	&
			   	-\mathbf{K} \frac{\Delta t}{6} 
		   	&
		   		\mathbf{K} \frac{\Delta t}{6} 
		\\ 	     
			   0 
			&
				\mathbf{K} 
		   	&
		   		-\mathbf{K} \frac{\Delta t}{2} 
		   	&
		   		-\mathbf{K} \frac{\Delta t}{2}
		\\   
		   		0
		   	& 
		   		0
		   	&
			   	\mathbf{K}
			   		\frac{(\Delta t)^2}{3} 
		   		+\mathbf{C} \frac{\Delta t}{2}
		   	&
		   		\mathbf{K} \frac{(\Delta t)^2}{6} 
		   		+\mathbf{M} 
			   	+\mathbf{C} \frac{\Delta t}{2}
		\\    
		   		0
		   	&
		   		0
		   	&
		   		\mathbf{K} \frac{(\Delta t)^2}{12}
		   		-\mathbf{M}
			   		\frac{1}{2} 
		   	&
		   		\mathbf{K} \frac{(\Delta t)^2}{12}
		   		+\mathbf{M} \frac{1}{6} 
			   +\mathbf{C} \frac{\Delta t}{6} 
	\end{bmatrix}
	\begin{bmatrix}
		   \mathbf{u}_m^+  		\\
		   \mathbf{u}_{m+1}^-  	\\
		   \mathbf{v}_m^+  		\\
		   \mathbf{v}_{m+1}^-  	\\
	\end{bmatrix}
	\\ =
	\begin{bmatrix}	
		  \mathbf{K} \mathbf{u}_m^-
		\\ \mathbf{K} \mathbf{u}_m^-
		\\ 	\left( \mathbf{M} \mathbf{v}_m^-
		     			+\frac{\Delta t}{2}  (\mathbf{f}_m + \mathbf{f}_{m+1})
			  \right)
			-\Delta t.
			 \left( \mathbf{K} \mathbf{u}_m^-
			  \right)
		\\-\frac{\Delta t}{6}
				\left( \mathbf{K} \mathbf{u}_m^- 
						-\mathbf{f}_{m+1}
				\right)
					  
			- \frac{1}{3} .  \mathbf{M} \mathbf{v}_m^-
	\end{bmatrix}
\end{array}
\end{equation}

\chapter{Formulation à variable séparée }

Note : Les indices de sommation à l'intérieur de $\Sigma$ ne sont pas indiqués pour éviter une surcharge des équations. Il s'agit de la somme des produits fournissant la solution PGD trouvée à l'itération précédente.

-Pertinence de la représentation des vitesses avec $\varphi$ ?

\begin{equation}
\begin{array}{c}
		\begin{bmatrix}   
		   		\mathbf{K}
			&
		   		0
		   	&
			   	-\mathbf{K} \frac{\Delta t}{6} 
		   	&
		   		\mathbf{K} \frac{\Delta t}{6} 
		\\ 	     
			   0 
			&
				\mathbf{K} 
		   	&
		   		-\mathbf{K} \frac{\Delta t}{2} 
		   	&
		   		-\mathbf{K} \frac{\Delta t}{2}
		\\   
		   		0
		   	& 
		   		0
		   	&
			   	\mathbf{K}
			   		\frac{(\Delta t)^2}{3} 
		   		+\mathbf{C} \frac{\Delta t}{2}
		   	&
		   		\mathbf{K} \frac{(\Delta t)^2}{6} 
		   		+\mathbf{M} 
			   	+\mathbf{C} \frac{\Delta t}{2}
		\\    
		   		0
		   	&
		   		0
		   	&
		   		\mathbf{K} \frac{(\Delta t)^2}{12}
		   		-\mathbf{M}
			   		\frac{1}{2} 
		   	&
		   		\mathbf{K} \frac{(\Delta t)^2}{12}
		   		+\mathbf{M} \frac{1}{6} 
			   +\mathbf{C} \frac{\Delta t}{6} 
	\end{bmatrix}
	\begin{bmatrix}
		   \varphi {\mathbf{g}^u_m}^+ h + \Sigma \varphi {\mathbf{g}^u_m}^+ h 		\\
		   \varphi {\mathbf{g}^u_{m+1}}^- h + \Sigma \varphi {\mathbf{g}^u_{m+1}}^ h  	\\
		   \varphi {\mathbf{g}^v_m}^+  h + \Sigma \varphi {\mathbf{g}^v_m}^+ h 		\\
		   \varphi {\mathbf{g}^v_{m+1}}^-  h + \Sigma \varphi {\mathbf{g}^v_{m+1}}^- h 	\\
	\end{bmatrix}
	\\ =
	\begin{bmatrix}	
		  \mathbf{K} (\varphi {\mathbf{g}^u_m}^- h + \Sigma \varphi {\mathbf{g}^u_m}^- h)
		\\ \mathbf{K} (\varphi {\mathbf{g}^u_m}^- h + \Sigma \varphi {\mathbf{g}^u_m}^-  h)
		\\ 	\left( \mathbf{M}  (\varphi {\mathbf{g}^v_m}^- h + \Sigma \varphi {\mathbf{g}^v_m}^- h)
		     			+\frac{\Delta t}{2}  (\mathbf{f}_m + \mathbf{f}_{m+1})
			  \right)
			-\Delta t.
			 \left( \mathbf{K}  (\varphi {\mathbf{g}^u_m}^- h + \Sigma \varphi {\mathbf{g}^u_m}^- h)
			  \right)
		\\-\frac{\Delta t}{6}
				\left( \mathbf{K}  (\varphi {\mathbf{g}^u_m}^- h + \Sigma \varphi {\mathbf{g}^u_m}^- h)
						-\mathbf{f}_{m+1}
				\right)
					  
			- \frac{1}{3} .  \mathbf{M} (\varphi {\mathbf{g}^v_m}^- h + \Sigma \varphi {\mathbf{g}^v_m}^- h)
	\end{bmatrix}
\end{array}
\end{equation}
 
\begin{equation}
\begin{array}{c}
		\begin{bmatrix}   
		   		\mathbf{K}
			&
		   		0
		   	&
			   	-\mathbf{K} \frac{\Delta t}{6} 
		   	&
		   		\mathbf{K} \frac{\Delta t}{6} 
		\\ 	     
			   0 
			&
				\mathbf{K} 
		   	&
		   		-\mathbf{K} \frac{\Delta t}{2} 
		   	&
		   		-\mathbf{K} \frac{\Delta t}{2}
		\\   
		   		0
		   	& 
		   		0
		   	&
			   	\mathbf{K}
			   		\frac{(\Delta t)^2}{3} 
		   		+\mathbf{C} \frac{\Delta t}{2}
		   	&
		   		\mathbf{K} \frac{(\Delta t)^2}{6} 
		   		+\mathbf{M} 
			   	+\mathbf{C} \frac{\Delta t}{2}
		\\    
		   		0
		   	&
		   		0
		   	&
		   		\mathbf{K} \frac{(\Delta t)^2}{12}
		   		-\mathbf{M}
			   		\frac{1}{2} 
		   	&
		   		\mathbf{K} \frac{(\Delta t)^2}{12}
		   		+\mathbf{M} \frac{1}{6} 
			   +\mathbf{C} \frac{\Delta t}{6} 
	\end{bmatrix}
	\left(
	\begin{bmatrix}
		\varphi h \begin{bmatrix}
		   {\mathbf{g}^u_m}^+  		\\
		   {\mathbf{g}^u_{m+1}}^-  	\\
		   {\mathbf{g}^v_m}^+  		\\
		   {\mathbf{g}^v_{m+1}}^-  	\\
		\end{bmatrix}
		+
		\begin{bmatrix}
		   \Sigma \varphi {\mathbf{g}^u_m}^+ h 		\\
		   \Sigma \varphi {\mathbf{g}^u_{m+1}}^- h  	\\
		   \Sigma \varphi {\mathbf{g}^v_m}^+ h 		\\
		   \Sigma \varphi {\mathbf{g}^v_{m+1}}^- h 	\\
		\end{bmatrix}
	\end{bmatrix}
	\right)
	\\ =
	\begin{bmatrix}	
		  \mathbf{K} (\varphi {\mathbf{g}^u_m}^- h + \Sigma \varphi {\mathbf{g}^u_m}^- h)
		\\ \mathbf{K} (\varphi {\mathbf{g}^u_m}^- h + \Sigma \varphi {\mathbf{g}^u_m}^- h)
		\\ 	\left( \mathbf{M}  (\varphi {\mathbf{g}^v_m}^- h + \Sigma \varphi {\mathbf{g}^v_m}^- h)
		     			+\frac{\Delta t}{2}  (\mathbf{f}_m + \mathbf{f}_{m+1})
			  \right)
			-\Delta t.
			 \left( \mathbf{K}  (\varphi {\mathbf{g}^u_m}^- h + \Sigma \varphi{\mathbf{g}^u_m}^- h)
			  \right)
		\\-\frac{\Delta t}{6}
				\left( \mathbf{K}  (\varphi {\mathbf{g}^u_m}^- h + \Sigma \varphi {\mathbf{g}^u_m}^- h)
						-\mathbf{f}_{m+1}
				\right)
					  
			- \frac{1}{3} .  \mathbf{M} (\varphi {\mathbf{g}^v_m}^- h + \Sigma \varphi {\mathbf{g}^v_m}^- h)
	\end{bmatrix}
\end{array}
\end{equation}
 
 L'équation doit être scalaire :
 
\begin{equation}
\!\!\!\!\!\!\!\!\!\!\!\!\!\!\!\!\!
\begin{array}{c}
	(\varphi h)^T \times
		\begin{bmatrix}   
		   		\mathbf{K}
			&
		   		0
		   	&
			   	-\mathbf{K} \frac{\Delta t}{6} 
		   	&
		   		\mathbf{K} \frac{\Delta t}{6} 
		\\ 	     
			   0 
			&
				\mathbf{K} 
		   	&
		   		-\mathbf{K} \frac{\Delta t}{2} 
		   	&
		   		-\mathbf{K} \frac{\Delta t}{2}
		\\   
		   		0
		   	& 
		   		0
		   	&
			   	\mathbf{K}
			   		\frac{(\Delta t)^2}{3} 
		   		+\mathbf{C} \frac{\Delta t}{2}
		   	&
		   		\mathbf{K} \frac{(\Delta t)^2}{6} 
		   		+\mathbf{M} 
			   	+\mathbf{C} \frac{\Delta t}{2}
		\\    
		   		0
		   	&
		   		0
		   	&
		   		\mathbf{K} \frac{(\Delta t)^2}{12}
		   		-\mathbf{M}
			   		\frac{1}{2} 
		   	&
		   		\mathbf{K} \frac{(\Delta t)^2}{12}
		   		+\mathbf{M} \frac{1}{6} 
			   +\mathbf{C} \frac{\Delta t}{6} 
	\end{bmatrix}
	\left(
	\begin{bmatrix}
		\varphi h \begin{bmatrix}
		   {\mathbf{g}^u_m}^+  		\\
		   {\mathbf{g}^u_{m+1}}^-  	\\
		   {\mathbf{g}^v_m}^+  		\\
		   {\mathbf{g}^v_{m+1}}^-  	\\
		\end{bmatrix}
		+
		\begin{bmatrix}
		   \Sigma \varphi {\mathbf{g}^u_m}^+ h 		\\
		   \Sigma \varphi {\mathbf{g}^u_{m+1}}^- h  	\\
		   \Sigma \varphi {\mathbf{g}^v_m}^+ h 		\\
		   \Sigma \varphi {\mathbf{g}^v_{m+1}}^- h 	\\
		\end{bmatrix}
	\end{bmatrix}
	\right)
	\\ =	
			(\varphi h)^T \times
	\begin{bmatrix}	
		  \mathbf{K} (\varphi {\mathbf{g}^u_m}^- h + \Sigma \varphi {\mathbf{g}^u_m}^- h)
		\\ \mathbf{K} (\varphi {\mathbf{g}^u_m}^- h + \Sigma \varphi {\mathbf{g}^u_m}^- h)
		\\ 	\left( \mathbf{M}  (\varphi {\mathbf{g}^v_m}^- h + \Sigma \varphi {\mathbf{g}^v_m}^- h)
		     			+\frac{\Delta t}{2}  (\mathbf{f}_m + \mathbf{f}_{m+1})
			  \right)
			-\Delta t.
			 \left( \mathbf{K}  (\varphi {\mathbf{g}^u_m}^- h + \Sigma \varphi{\mathbf{g}^u_m}^- h)
			  \right)
		\\-\frac{\Delta t}{6}
				\left( \mathbf{K}  (\varphi {\mathbf{g}^u_m}^- h + \Sigma \varphi {\mathbf{g}^u_m}^- h)
						-\mathbf{f}_{m+1}
				\right)
					  
			- \frac{1}{3} .  \mathbf{M} (\varphi {\mathbf{g}^v_m}^- h + \Sigma \varphi {\mathbf{g}^v_m}^- h)
	\end{bmatrix}
\end{array}
\end{equation}
 
\begin{equation}
\!\!\!\!\!\!\!\!\!\!\!\!\!\!\!\!\!
\begin{array}{c}
	(\varphi h)^T \times
		\begin{bmatrix}   
		   		\mathbf{K}
			&
		   		0
		   	&
			   	-\mathbf{K} \frac{\Delta t}{6} 
		   	&
		   		\mathbf{K} \frac{\Delta t}{6} 
		\\ 	     
			   0 
			&
				\mathbf{K} 
		   	&
		   		-\mathbf{K} \frac{\Delta t}{2} 
		   	&
		   		-\mathbf{K} \frac{\Delta t}{2}
		\\   
		   		0
		   	& 
		   		0
		   	&
			   	\mathbf{K}
			   		\frac{(\Delta t)^2}{3} 
		   		+\mathbf{C} \frac{\Delta t}{2}
		   	&
		   		\mathbf{K} \frac{(\Delta t)^2}{6} 
		   		+\mathbf{M} 
			   	+\mathbf{C} \frac{\Delta t}{2}
		\\    
		   		0
		   	&
		   		0
		   	&
		   		\mathbf{K} \frac{(\Delta t)^2}{12}
		   		-\mathbf{M}
			   		\frac{1}{2} 
		   	&
		   		\mathbf{K} \frac{(\Delta t)^2}{12}
		   		+\mathbf{M} \frac{1}{6} 
			   +\mathbf{C} \frac{\Delta t}{6} 
	\end{bmatrix}
	\left(
		\varphi h \begin{bmatrix}
		   {\mathbf{g}^u_m}^+  		\\
		   {\mathbf{g}^u_{m+1}}^-  	\\
		   {\mathbf{g}^v_m}^+  		\\
		   {\mathbf{g}^v_{m+1}}^-  	\\
		\end{bmatrix}
	\right)
	\\ =	
			(\varphi h)^T \times
	\begin{bmatrix}	
		  \mathbf{K} (\varphi {\mathbf{g}^u_m}^- h + \Sigma \varphi {\mathbf{g}^u_m}^- h)
		\\ \mathbf{K} (\varphi {\mathbf{g}^u_m}^- h + \Sigma \varphi {\mathbf{g}^u_m}^- h)
		\\ 	\left( \mathbf{M}  (\varphi {\mathbf{g}^v_m}^- h + \Sigma \varphi {\mathbf{g}^v_m}^- h)
		     			+\frac{\Delta t}{2}  (\mathbf{f}_m + \mathbf{f}_{m+1})
			  \right)
			-\Delta t.
			 \left( \mathbf{K}  (\varphi {\mathbf{g}^u_m}^- h + \Sigma \varphi{\mathbf{g}^u_m}^- h)
			  \right)
		\\-\frac{\Delta t}{6}
				\left( \mathbf{K}  (\varphi {\mathbf{g}^u_m}^- h + \Sigma \varphi {\mathbf{g}^u_m}^- h)
						-\mathbf{f}_{m+1}
				\right)
					  
			- \frac{1}{3} .  \mathbf{M} (\varphi {\mathbf{g}^v_m}^- h + \Sigma \varphi {\mathbf{g}^v_m}^- h)
	\end{bmatrix}
	\\
	-
	(\varphi h)^T \times
		\begin{bmatrix}   
		   		\mathbf{K}
			&
		   		0
		   	&
			   	-\mathbf{K} \frac{\Delta t}{6} 
		   	&
		   		\mathbf{K} \frac{\Delta t}{6} 
		\\ 	     
			   0 
			&
				\mathbf{K} 
		   	&
		   		-\mathbf{K} \frac{\Delta t}{2} 
		   	&
		   		-\mathbf{K} \frac{\Delta t}{2}
		\\   
		   		0
		   	& 
		   		0
		   	&
			   	\mathbf{K}
			   		\frac{(\Delta t)^2}{3} 
		   		+\mathbf{C} \frac{\Delta t}{2}
		   	&
		   		\mathbf{K} \frac{(\Delta t)^2}{6} 
		   		+\mathbf{M} 
			   	+\mathbf{C} \frac{\Delta t}{2}
		\\    
		   		0
		   	&
		   		0
		   	&
		   		\mathbf{K} \frac{(\Delta t)^2}{12}
		   		-\mathbf{M}
			   		\frac{1}{2} 
		   	&
		   		\mathbf{K} \frac{(\Delta t)^2}{12}
		   		+\mathbf{M} \frac{1}{6} 
			   +\mathbf{C} \frac{\Delta t}{6} 
	\end{bmatrix}
		\begin{bmatrix}
		   \Sigma \varphi {\mathbf{g}^u_m}^+ h 		\\
		   \Sigma \varphi {\mathbf{g}^u_{m+1}}^- h  	\\
		   \Sigma \varphi {\mathbf{g}^v_m}^+ h 		\\
		   \Sigma \varphi {\mathbf{g}^v_{m+1}}^- h 	\\
		\end{bmatrix}
\end{array}
\end{equation}

\begin{equation}
\begin{array}{c}
	\\
	 \varphi h^T \times \mathbf{f} 		= f
	\\
	\varphi h^T \times \mathbf{K} \times \varphi h			= K
	\\
	\varphi h^T \times \mathbf{K} \times \varphi_i h_i	= K_i
\end{array}
\end{equation}
	Où $_i$ est l'indice de sommation.  Même chose pour $\mathbf{M}$ et $\mathbf{C}$.
	Ceci permet de n'avoir plus que des scalaires.
 
\begin{equation}
\!\!\!\!\!\!\!\!\!\!\!\!\!\!\!\!\!
\begin{array}{c}
		\begin{bmatrix}   
		   		K  &  0  &	-K \frac{\Delta t}{6}   &  K \frac{\Delta t}{6} 
		\\ 	     
			   0  &  K	 &  -K \frac{\Delta t}{2} 	&  -K \frac{\Delta t}{2}
		\\   
		   		0  &  0  &  	K
			   						\frac{(\Delta t)^2}{3} 
		   							+C \frac{\Delta t}{2}
		   	&
		   		K \frac{(\Delta t)^2}{6} 
		   		+M 
			   	+C \frac{\Delta t}{2}
		\\    
		   		0 	&  0  &  K \frac{(\Delta t)^2}{12}
						   		-M \frac{1}{2} 
		   	&
		   		K \frac{(\Delta t)^2}{12}
		   		+M \frac{1}{6} 
			   +C \frac{\Delta t}{6} 
	\end{bmatrix}
		\begin{bmatrix}
		   {\mathbf{g}^u_m}^+  		\\
		   {\mathbf{g}^u_{m+1}}^-  	\\
		   {\mathbf{g}^v_m}^+  		\\
		   {\mathbf{g}^v_{m+1}}^-  	\\
		\end{bmatrix}
	\\ =	
	\begin{bmatrix}	
		  K {\mathbf{g}^u_m}^- + \Sigma K_i ({\mathbf{g}^u_m}^-)_i
		\\ K {\mathbf{g}^u_m}^- + \Sigma K_i ({\mathbf{g}^u_m}^-)_i
		\\ 	 M {\mathbf{g}^v_m}^- + \Sigma M_i ({\mathbf{g}^v_m}^-)_i
		     			+\frac{\Delta t}{2}  (f_m + f_{m+1})
			-\Delta t
			 \left(  K {\mathbf{g}^u_m}^- + \Sigma K_i ({\mathbf{g}^u_m}^-)_i 
			  \right)
		\\-\frac{\Delta t}{6}
				\left( K {\mathbf{g}^u_m}^- + \Sigma K_i ({\mathbf{g}^u_m}^- )_i )
						-f_{m+1}
				\right)
			- \frac{1}{3}  (M {\mathbf{g}^v_m}^- + \Sigma K_i ({\mathbf{g}^v_m}^-)_i )
	\end{bmatrix}
	\\
	- \displaystyle \sum^i
		\begin{bmatrix}   
		   		K_i  &  0  &  -K_i \frac{\Delta t}{6}  &  K_i \frac{\Delta t}{6} 
		\\ 	     
			   0  &  K_i  &  -K_i \frac{\Delta t}{2}  &  -K_i \frac{\Delta t}{2}
		\\   
		   		0  &  0  &  K_i
						   		\frac{(\Delta t)^2}{3} 
					   			+C_i \frac{\Delta t}{2}
		   	&
		   		K_i \frac{(\Delta t)^2}{6} 
		   		+M_i 
			   	+C_i \frac{\Delta t}{2}
		\\    
		   		0  &  0  &  K_i \frac{(\Delta t)^2}{12}
		   						-M_i \frac{1}{2} 
		   	&
		   		K_i \frac{(\Delta t)^2}{12}
		   		+M_i \frac{1}{6} 
			   +C_i \frac{\Delta t}{6} 
	\end{bmatrix}
		\begin{bmatrix}
		  {\mathbf{g}^u_m}^+ 		\\
		  {\mathbf{g}^u_{m+1}}^-  	\\
		  {\mathbf{g}^v_m}^+ 		\\
		  {\mathbf{g}^v_{m+1}}^- 	\\
		\end{bmatrix}_i
\end{array}
\end{equation}

\end{document}
