On désire calculer les intégrales pour résoudre les problèmes comme celui en espace :
\begin{equation}
\begin{array}{r r l}
	&& \displaystyle
		\int_T \! \int_\Theta
			gh [  ~K~ gh
				+ ~C~ \dot{g}h 
				+ ~M~ \ddot{g}h
				] ~dt d\theta 
	~ f_q
	\\
	= &-& \displaystyle
		\int_T \! \int_\Theta		
			gh [  K~ \sum_{k=1}^{n-1} (f_q)_k       g_k  h_k 
				+ ~C~ \sum_{k=1}^{n-1} (f_q)_k  \dot{g_k} h_k 
				+ ~M~ \sum_{k=1}^{n-1} (f_q)_k \ddot{g_k} h_k
				- ~F~] ~dt d\theta
\end{array}
\end{equation}

On peut obtenir des termes développés ainsi:
\begin{equation}
\begin{array}{r r l}

	&&\left[\displaystyle
	\int_T \! \int_\Theta
		g^2h^2 ~dt d\theta ~K~
	+ \int_T \! \int_\Theta
		g \dot{g} h^2 ~dt d\theta ~C~ 
	+ \int_T \! \int_\Theta
		g \ddot{g} h^2 ~dt d\theta ~M~
			\right] ~ f_q
	\\
	= &-&
	 \displaystyle
	 \int_T \! \int_\Theta		
		gh \sum_{k=1}^{n-1} (f_q)_k       g_k  h_k ~K~ ~dt d\theta
	\\
	&-&
	 \displaystyle
	 \int_T \! \int_\Theta
	 	gh \sum_{k=1}^{n-1} (f_q)_k  \dot{g_k} h_k ~C~  ~dt d\theta
	\\
	&-&
	 \displaystyle
	 \int_T \! \int_\Theta
	 	gh \sum_{k=1}^{n-1} (f_q)_k \ddot{g_k} h_k ~M~ ~dt d\theta
	\\
	&+&
	 \displaystyle
	 \int_T \! \int_\Theta
	 	gh ~F ~dt d\theta 
\end{array}
\end{equation}

En inversant somme et intégrale on a :
\begin{equation}
\begin{array}{r r l}

	&&\left[\displaystyle
	\int_T \! \int_\Theta
		g^2h^2 ~dt d\theta ~K~
	+ \int_T \! \int_\Theta
		g \dot{g} h^2 ~dt d\theta ~C~ 
	+ \int_T \! \int_\Theta
		g \ddot{g} h^2 ~dt d\theta ~M~
			\right] ~ f_q
	\\
	= &-&
	 \displaystyle 
	 \sum_{k=1}^{n-1}
		 \int_T \! \int_\Theta		
			gh (f_q)_k       g_k  h_k ~K~ ~dt d\theta
	\\
	&-&
	 \displaystyle
	 \sum_{k=1}^{n-1}
		 \int_T \! \int_\Theta
		 	gh (f_q)_k  \dot{g_k} h_k ~C~  ~dt d\theta
	\\
	&-&
	 \displaystyle
	 \sum_{k=1}^{n-1}
		 \int_T \! \int_\Theta
		 	gh (f_q)_k \ddot{g_k} h_k ~M~ ~dt d\theta
	\\
	&+&
	 \displaystyle
	 \int_T \! \int_\Theta
	 	gh ~F ~dt d\theta 
\end{array}
\end{equation}

En faisant l'hypothèse que les matrices ne dépendent pas des paramètres intégrés (hypothèse fausse en non-linéaire), la séparation des variables nous donne :
\begin{equation}
\begin{array}{r r l}

	&&\left[\displaystyle
	\int_T g^2~dt 			\! \int_\Theta
		h^2  d\theta ~K~
	+ \int_T g \dot{g}~dt 	\! \int_\Theta
		 h^2  d\theta ~C~ 
	+ \int_T g \ddot{g}~dt 	\! \int_\Theta
		 h^2  d\theta ~M~
			\right] ~ f_q
	\\
	= &-&
	 \displaystyle 
	 \sum_{k=1}^{n-1}
		 \int_T g g_k dt \! \int_\Theta		
			 h h_k d\theta  ~(f_q)_k ~K~
	\\
	&-&
	 \displaystyle
	 \sum_{k=1}^{n-1}
		 \int_T g \dot{g_k} dt \! \int_\Theta
		 	h h_k  d\theta  ~(f_q)_k ~C~
	\\
	&-&
	 \displaystyle
	 \sum_{k=1}^{n-1}
		 \int_T g \ddot{g_k} dt \! \int_\Theta
		 	h h_k d\theta  ~(f_q)_k ~M~
	\\
	&+&
	 \displaystyle
	 \int_T g ~F dt \! \int_\Theta
	 	h d\theta 
\end{array}
\end{equation}

On voit que une intégrale simple d'un produit de deux fonctions est un élément calculer de nombreuse fois, il faut donc en créer un programme.

\section{Intégration d'un produit de fonctions linéaires par morceaux}

Les calculs sont évidemment effectués sur des objet discrétiser pour pouvoir utiliser des outils numériques. Alors on considère deux vecteurs $G_1$ et $G_2$ contenant chacun les valeurs d'une fonction discrétisée, respectivement $g_1(t)$ et $g_2(t)$, définie $\forall t \in [t_0;t_0+T]$ tels que:  
\begin{equation}
	\forall n \in [1;(T/ \delta t)] ~G_1(n) = g_1((n-1) \delta t+t_0)
	~~	idem~pour~G_2
\end{equation}

Il faut alors choisir une méthode de calcul d'intégrale, la première présentée ici fais l'hypothèse que les fonction sont linéaires entre les valeurs discrètes connues.

\begin{equation}
	\int_T g_1(t) g_2(t) dt = \sum_{n=1}^{T/\delta t} 
							\int_{t_0+(n-1)\delta t}^{t_0+n\delta t} 
								g_1(t) g_2(t) dt 
\end{equation}

Et si on applique l'hypothèse de linéarité, on peut écrire localement (sur le domaine de l'intégrale précédente) $g_1$ sous la forme : $at+b$, et $g_2$ sous la forme : $et+f$, où $a$, $b$, $e$ et $f$ sont des réels. Pour alléger la notation on définit $t_1=t_0+(n-1)\delta t$ et $t_2=t_0+n\delta t$, alors :
\begin{equation}
\begin{array}{r c}
	&\displaystyle
		\int_{t_1}^{t_2} g_1(t) g_2(t) dt
	\\
	=&\displaystyle
		\int_{t_1}^{t_2} (at+b)(et+f) dt
	\\
	=&\displaystyle
		\int_{t_1}^{t_2} \left(a e t^2 + (a f + b e) t + b f \right) dt
	\\
	=&\displaystyle
		\left[ a e \frac{t^3}{3} 
				+ (a f + b e) \frac{t^2}{2} 
				+ b f t \right]_{t_1}^{t_2}
	\\
	=&\displaystyle
		a e \frac{t_2^3 - t_1^3}{3} 
				+ (a f + b e) \frac{t_2^2 - t_1^2}{2} 
				+ b f (t_2 - t_1)
\end{array} 
\end{equation}

Donc :
\begin{equation}
	\int_T g_1(t) g_2(t) dt = \sum_{n=1}^{T/dt} 
							\left[ a e \frac{t_2^3 - t_1^3}{3} 
				+ (a f + b e) \frac{t_2^2 - t_1^2}{2} 
				+ b f (t_2 - t_1) \right]
\end{equation}

On peut identifier :
\begin{equation}
\begin{array}{r c c c l}
		a &=& \displaystyle \frac{g_1(t_1)-g_1(t_2)}{\delta t}
		  &=& \displaystyle \frac{G_1(n)  -G_1(n+1)}{\delta t}
	\\
		b &=& g_1(t_1) - at_1
		  &=& G_1(n) - at_1
	\\
		&&& $et identiquement$ &
	\\
		e &=& \displaystyle \frac{g_2(t_1)-g_2(t_2)}{\delta t}
		  &=& \displaystyle \frac{G_2(n)  -G_2(n+1)}{\delta t}
	\\
		f &=& g_2(t_1) - et_1
		  &=& G_2(n) - et_1
	
\end{array}
\end{equation}


\section{Intégration de produit de fonctions constantes par morceaux}

Si on considère les fonctions sont constantes par morceaux, on peut écrire l'intégrale sous la forme : 
\begin{equation}
	\int_T g_1(t) g_2(t) dt 	= \sum_{n=1}^{T/\delta t} 
							\int_{t_0+(n-1)\delta t}^{t_0+n\delta t} 
								g_1(t) g_2(t) dt 
						= \sum_{n=1}^{T/\delta t} 
							G_1(n) G_2(n) dt 
\end{equation}

Ceci correspond à une intégrale de Riemann, et on peut donc en avoir quelques variantes en changeant les indices de sommation :
\begin{equation}
\begin{array}{c}
		\displaystyle
		\sum_{n=1}^{T/\delta t} 
							G_1(n) G_2(n) dt 
	\\
		\displaystyle
		\sum_{n=2}^{T/\delta t +1} 
							G_1(n) G_2(n) dt 
	\\
		\displaystyle
		\sum_{n=1}^{T/\delta t +1} 
							G_1(n) G_2(n) dt 
		= G_1.G_2 dt
\end{array}
\end{equation}