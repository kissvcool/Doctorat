%\documentclass[10pt]{report}
%
%\begin{document}
\begin{itemize}
\item Admissibilité cinématique :
	\\$U \in [H^1(\Omega)]^3, U_{|\partial \Omega_U} = U_d $
\item Admissibilité statique :
	\\$ div \sigma + f_v = \rho \ddot{U} + \mu \dot{U}$ sur $\Omega$ 
		et $\sigma . n _{|\partial \Omega_f} = f_d$
\item Relation de comportement :
	\\$\sigma = H : \epsilon (U)$ sur $ \Omega$
\item Conditions initiales :
	\\$U(0) = U_0$ et $ \dot{U}(0) = \dot{U}_0$
\end{itemize}

\noindent
Formulation variationnelle : 
\begin{equation}
	\forall U^* ~ \text{CA0,} ~
		\int_\Omega [\epsilon (U^*) : \sigma   
					+ U^* \mu \dot{U} 
					+ U^* \rho \ddot{U}
					- U^* f_d
					] d \Omega	
		- \int_{\partial \Omega_f} U^* f_d  ds		
		- \underbrace{
			\int_{\partial \Omega_U} U^* \sigma . n ds	
		  }_{=0}
		= 0
\end{equation}

\noindent
Où
\begin{itemize}
\item $U$ est le déplacement
\item $\Omega$ l'espace, avec le bord $\partial \Omega = \partial \Omega_U \cap \partial \Omega_f$
\item $\sigma$ le tenseur des contraintes
\item $\epsilon$ le tenseur des déformations
\item $\rho$ la masse volumique
\item $\mu$ expression de la viscosité (cette donnée est généralement mal connue)
\item $H$ le tenseur de Hook
\item $f_v$ effort volumique
\item $f_d$ effort imposé que le bord de $\Omega$
\end{itemize}
\vspace{0.3cm}

\noindent
Mise en place de la PGD, utilisation de l'hypothèse des variables séparables :
\begin{equation}
	U(x,t,\theta) = \sum_{k=1}^n \phi_k(x) g_k(t)h_k(\theta)
\end{equation}

Pour chaque $k$, il faut discrétiser les fonctions sur leur domaine de définition : 
\begin{equation}
	\begin{array}{l c l c l}
		\phi(x) &=& \sum_{i=1}^{Nbc_x}  N_i (x) \phi_i &=& N_\phi(x) \boldsymbol{\phi_q}
		\\
		g(t) &=& \sum_{i=1}^{Nbc_t}  N_i (t) gi &=& N_g(t) \mathbf{g_q}
		\\
		h(\theta) &=& \sum_{i=1}^{Nbc_\theta}  N_i (\theta) hi 
		&=& N_h(\theta) \mathbf{h_q}
	\end{array}
\end{equation}
$Nbc_x$ Représente le nombre de composante de discrétisation spatiale, l'indice $q$ représente le vecteur sur le domaine discrétisé.
\\
On remplace dans la formulation variationnelle, en choisissant $U^* = (\phi gh)^* = ( \phi^*gh+\phi g^*h+\phi gh^*) $ :
\begin{equation}
\begin{array}{r r l}
	&\int_\Omega \! \int_T \! \int_\Theta \!\!\!&		
		[\epsilon (\phi^*gh) : H : \epsilon (U_n)
			+ \phi^*gh \mu \dot{U_n} 
			+ \phi^*gh \rho \ddot{U_n}
			- \phi^*gh f_v
			] ~dx dt d\theta
	\\
	+ &\int_\Omega \! \int_T \! \int_\Theta \!\!\!&		
		[\epsilon (\phi g^*h) : H : \epsilon (U_n)
			+ \phi g^*h \mu \dot{U_n} 
			+ \phi g^*h \rho \ddot{U_n}
			- \phi g^*h f_v
			] ~dx dt d\theta
	\\
	+ &\int_\Omega \! \int_T \! \int_\Theta \!\!\!&		
		[\epsilon (\phi gh^*) : H : \epsilon (U_n)
			+ \phi gh^* \mu \dot{U_n} 
			+ \phi gh^* \rho \ddot{U_n}
			- \phi gh^* f_v
			] ~dx dt d\theta
	\\
	- &\int_{\partial \Omega_\mathbf{f}} \! \int_T \! \int_\Theta \!\!\!&
		(\phi^*gh + \phi g^*h + \phi gh^*) f_d  ~ds dt d\theta
\\
	= &0& 
\end{array}
\end{equation}

Puis $U = U_{n} = U_{n-1} + \phi gh$ :
\begin{equation}
\begin{array}{r r l}
	&\int_\Omega \! \int_T \! \int_\Theta \!\!\!&		
		gh[\epsilon (\phi^*) : H : \epsilon (U_{n-1} + \phi gh)
			+ \phi^* \mu \dot{(U_{n-1} + \phi gh)} 
			+ \phi^* \rho \ddot{(U_{n-1} + \phi gh)}
			- \phi^* f_v
			] ~dx dt d\theta
	\\
	+ &\int_\Omega \! \int_T \! \int_\Theta \!\!\!&		
		g^*h[\epsilon (\phi) : H : \epsilon (U_{n-1} + \phi gh)
			+ \phi \mu \dot{(U_{n-1} + \phi gh)} 
			+ \phi \rho \ddot{(U_{n-1} + \phi gh)}
			- \phi f_v
			] ~dx dt d\theta
	\\
	+ &\int_\Omega \! \int_T \! \int_\Theta \!\!\!&		
		gh^*[\epsilon (\phi) : H : \epsilon (U_{n-1} + \phi gh)
			+ \phi \mu \dot{(U_{n-1} + \phi gh)} 
			+ \phi \rho \ddot{(U_{n-1} + \phi gh)}
			- \phi f_v
			] ~dx dt d\theta
	\\
	- &\int_{\partial \Omega_\mathbf{f}} \! \int_T \! \int_\Theta \!\!\!&
		(\phi^*gh + \phi g^*h + \phi gh^*) f_d  ~ds dt d\theta
\\
	= &0& 
\end{array}
\end{equation}

On distribue la dérivation : 
\begin{equation}
\begin{array}{r r l}
	&\int_\Omega \! \int_T \! \int_\Theta \!\!\!&		
		gh[\epsilon (\phi^*) : H : \epsilon (U_{n-1} + \phi gh)
			+ \phi^* \mu (\dot{U_{n-1}} + \phi \dot{g}h)
			+ \phi^* \rho (\ddot{U_{n-1}} + \phi \ddot{g}h)
			- \phi^* f_v
			] ~dx dt d\theta
	\\
	+ &\int_\Omega \! \int_T \! \int_\Theta \!\!\!&		
		g^*h[\epsilon (\phi) : H : \epsilon (U_{n-1} + \phi gh)
			+ \phi \mu (\dot{U_{n-1}} + \phi \dot{g}h) 
			+ \phi \rho (\ddot{U_{n-1}} + \phi \ddot{g}h)
			- \phi f_v
			] ~dx dt d\theta
	\\
	+ &\int_\Omega \! \int_T \! \int_\Theta \!\!\!&		
		gh^*[\epsilon (\phi) : H : \epsilon (U_{n-1} + \phi gh)
			+ \phi \mu (\dot{U_{n-1}} + \phi\dot{g}h)
			+ \phi \rho (\ddot{U_{n-1}} + \phi\ddot{g}h)
			- \phi f_v
			] ~dx dt d\theta
	\\
	- &\int_{\partial \Omega_\mathbf{f}} \! \int_T \! \int_\Theta \!\!\!&
		(\phi^*gh + \phi g^*h + \phi gh^*) f_d  ~ds dt d\theta
\\
	= &0& 
\end{array}
\end{equation}

On sépare les termes en $n-1$ des termes virtuels :
\begin{equation}
\begin{array}{r r l}
	&\int_\Omega \! \int_T \! \int_\Theta \!\!\!&		
		gh[\epsilon (\phi^*) : H : \epsilon (\phi gh)
			+ \phi^* \mu \phi\dot{g}h
			+ \phi^* \rho \phi\ddot{g}h
			- \phi^* f_v
			] ~dx dt d\theta
	\\
	+ &\int_\Omega \! \int_T \! \int_\Theta \!\!\!&		
		gh[\epsilon (\phi^*) : H : \epsilon (U_{n-1})
			+ \phi^* \mu \dot{U_{n-1}} 
			+ \phi^* \rho \ddot{U_{n-1}}
			] ~dx dt d\theta
	\\
	+ &\int_\Omega \! \int_T \! \int_\Theta \!\!\!&		
		g^*h[\epsilon (\phi) : H : \epsilon ( \phi gh)
			+ \phi \mu  \phi\dot{g}h
			+ \phi \rho \phi\ddot{g}h
			- \phi f_v
			] ~dx dt d\theta
	\\
	+ &\int_\Omega \! \int_T \! \int_\Theta \!\!\!&		
		g^*h[\epsilon (\phi) : H : \epsilon (U_{n-1})
			+ \phi \mu \dot{U_{n-1}}
			+ \phi \rho \ddot{U_{n-1}}
			] ~dx dt d\theta
	\\
	+ &\int_\Omega \! \int_T \! \int_\Theta \!\!\!&		
		gh^*[\epsilon (\phi) : H : \epsilon (\phi gh)
			+ \phi \mu + \phi\dot{g}h
			+ \phi \rho \phi\ddot{g}h
			- \phi f_v
			] ~dx dt d\theta
	\\
	+ &\int_\Omega \! \int_T \! \int_\Theta \!\!\!&		
		gh^*[\epsilon (\phi) : H : \epsilon (U_{n-1})
			+ \phi \mu \dot{U_{n-1}}
			+ \phi \rho \ddot{U_{n-1}}
			- \phi f_v
			] ~dx dt d\theta
	\\
	- &\int_{\partial \Omega_\mathbf{f}} \! \int_T \! \int_\Theta \!\!\!&
		(\phi^*gh + \phi g^*h + \phi gh^*) f_d  ~ds dt d\theta
	\\
	= &0& 
\end{array}
\end{equation}

On peut regrouper les termes qui n'ont pas de partie virtuel en espace i.e. $\phi^*$ :
\begin{equation}
\begin{array}{r r l}
	&\int_\Omega \! \int_T \! \int_\Theta \!\!\!&		
		gh[\epsilon (\phi^*) : H : \epsilon (\phi gh)
			+ \phi^* \mu \phi\dot{g}h
			+ \phi^* \rho \phi\ddot{g}h
			- \phi^* f_v
			] ~dx dt d\theta
	\\
	+ &\int_\Omega \! \int_T \! \int_\Theta \!\!\!&		
		gh[\epsilon (\phi^*) : H : \epsilon (U_{n-1})
			+ \phi^* \mu \dot{U_{n-1}} 
			+ \phi^* \rho \ddot{U_{n-1}}
			] ~dx dt d\theta
	\\
	+ &\int_\Omega \! \int_T \! \int_\Theta \!\!\!&		
		(g^*h + gh^*)[\epsilon (\phi) : H : \epsilon ( \phi gh)
			+ \phi \mu  \phi\dot{g}h
			+ \phi \rho \phi\ddot{g}h
			- \phi f_v
			] ~dx dt d\theta
	\\
	+ &\int_\Omega \! \int_T \! \int_\Theta \!\!\!&		
		(g^*h + gh^*)[\epsilon (\phi) : H : \epsilon (U_{n-1})
			+ \phi \mu \dot{U_{n-1}}
			+ \phi \rho \ddot{U_{n-1}}
			] ~dx dt d\theta
	\\
	- &\int_{\partial \Omega_\mathbf{f}} \! \int_T \! \int_\Theta \!\!\!&
		(\phi^*gh + \phi g^*h + \phi gh^*) f_d  ~ds dt d\theta
	\\
	= &0& 
\end{array}
\end{equation}

On discrétise en espace :
\begin{equation}
\begin{array}{r r l}
	&\int_\Omega \! \int_T \! \int_\Theta \!\!\!&		
		gh\boldsymbol{\phi_q}^*[\epsilon (N_\phi(x)) : H : \epsilon (N_\phi(x)\boldsymbol{\phi_q} gh)
			+ N_\phi(x) \mu  N_\phi(x) \boldsymbol{\phi_q}  \dot{g}h
			\\ && \phantom{gh\boldsymbol{\phi_q}^*[ }
			+ N_\phi(x) \rho N_\phi(x) \boldsymbol{\phi_q} \ddot{g}h
			- N_\phi(x) f_v
			] ~dx dt d\theta
	\\
	+ &\int_\Omega \! \int_T \! \int_\Theta \!\!\!&		
		gh\boldsymbol{\phi_q}^*[\epsilon (N_\phi(x)) : H : \epsilon (U_{n-1})
			+ N_\phi(x) \mu \dot{U_{n-1}} 
			+ N_\phi(x) \rho \ddot{U_{n-1}}
			] ~dx dt d\theta
	\\
	+ &\int_\Omega \! \int_T \! \int_\Theta \!\!\!&		
		(g^*h + gh^*)\boldsymbol{\phi_q}[\epsilon (N_\phi(x)) : H : \epsilon (N_\phi(x) \boldsymbol{\phi_q}gh)
			+ N_\phi(x) \mu  N_\phi(x) \boldsymbol{\phi_q} \dot{g}h
			\\ && \phantom{(g^*h + gh^*)\boldsymbol{\phi_q}[ }
			 + N_\phi(x) \rho N_\phi(x) \boldsymbol{\phi_q} \ddot{g}h
			- N_\phi(x) f_v
			] ~dx dt d\theta
	\\
	+ &\int_\Omega \! \int_T \! \int_\Theta \!\!\!&		
		(g^*h + gh^*)\boldsymbol{\phi_q}[\epsilon (N_\phi(x)) : H : \epsilon (U_{n-1})
			+ N_\phi(x) \mu \dot{U_{n-1}}
			+ N_\phi(x) \rho \ddot{U_{n-1}}
			] ~dx dt d\theta
	\\
	- &\int_{\partial \Omega_\mathbf{f}} \! \int_T \! \int_\Theta \!\!\!&
		N_\phi(x)(\boldsymbol{\phi_q}^*gh + \boldsymbol{\phi_q} g^*h + \boldsymbol{\phi_q} gh^*) f_d  ~ds dt d\theta
	\\
	= &0& 
\end{array}
\end{equation}

\begin{equation}
U_{n-1}(x,t,\theta)
	 = \sum_{k=1}^{n-1} \phi_k(x) g_k(t)h_k(\theta)
	 = \sum_{k=1}^{n-1} N_\phi(x)(\boldsymbol{\phi_q})_k g_k(t)h_k(\theta)
\end{equation}

En utilisant la formule précédente :
\begin{equation}
\begin{array}{r r l}
	&\int_\Omega \! \int_T \! \int_\Theta \!\!\!&		
		gh\boldsymbol{\phi_q}^*[\epsilon (N_\phi(x)) : H : \epsilon (N_\phi(x)\boldsymbol{\phi_q} gh)
			\\ && \phantom{gh\boldsymbol{\phi_q}^*[ }
			+ N_\phi(x) \mu  N_\phi(x) \boldsymbol{\phi_q}  \dot{g}h
			\\ && \phantom{gh\boldsymbol{\phi_q}^*[ }
			+ N_\phi(x) \rho N_\phi(x) \boldsymbol{\phi_q} \ddot{g}h
			- N_\phi(x) f_v
			] ~dx dt d\theta
	\\
	+ &\int_\Omega \! \int_T \! \int_\Theta \!\!\!&		
		gh\boldsymbol{\phi_q}^*[\epsilon (N_\phi(x)) : H : 
				\epsilon (\sum_{k=1}^{n-1} N_\phi(x)(\boldsymbol{\phi_q})_k g_k h_k)
			\\ && \phantom{gh\boldsymbol{\phi_q}^*[}
			+ N_\phi(x) \mu \sum_{k=1}^{n-1} N_\phi(x)(\boldsymbol{\phi_q})_k \dot{g_k} h_k 
			\\ && \phantom{gh\boldsymbol{\phi_q}^*[}
			+ N_\phi(x) \rho \sum_{k=1}^{n-1} N_\phi(x)(\boldsymbol{\phi_q})_k \ddot{g_k} h_k 
			] ~dx dt d\theta
	\\
	+ &\int_\Omega \! \int_T \! \int_\Theta \!\!\!&		
		(g^*h + gh^*)\boldsymbol{\phi_q}[\epsilon (N_\phi(x)) : H : \epsilon (N_\phi(x) \boldsymbol{\phi_q}gh)
			\\ && \phantom{(g^*h + gh^*)\boldsymbol{\phi_q}[ }
			+ N_\phi(x) \mu  N_\phi(x) \boldsymbol{\phi_q} \dot{g}h
			\\ && \phantom{(g^*h + gh^*)\boldsymbol{\phi_q}[ }
			 + N_\phi(x) \rho N_\phi(x) \boldsymbol{\phi_q} \ddot{g}h
			- N_\phi(x) f_v
			] ~dx dt d\theta
	\\
	+ &\int_\Omega \! \int_T \! \int_\Theta \!\!\!&
		(g^*h + gh^*)\boldsymbol{\phi_q}[\epsilon (N_\phi(x)) : H : 
					\epsilon (\sum_{k=1}^{n-1} N_\phi(x)(\boldsymbol{\phi_q})_k \dot{g_k} h_k )
			\\ && \phantom{(g^*h + gh^*)\boldsymbol{\phi_q}[}
			+ N_\phi(x) \mu \sum_{k=1}^{n-1} N_\phi(x)(\boldsymbol{\phi_q})_k \dot{g_k} h_k 
			\\ && \phantom{(g^*h + gh^*)\boldsymbol{\phi_q}[}
			+ N_\phi(x) \rho \sum_{k=1}^{n-1} N_\phi(x)(\boldsymbol{\phi_q})_k \ddot{g_k} h_k 
			] ~dx dt d\theta
	\\
	- &\int_{\partial \Omega_\mathbf{f}} \! \int_T \! \int_\Theta \!\!\!&
		N_\phi(x)(\boldsymbol{\phi_q}^*gh + \boldsymbol{\phi_q} g^*h + \boldsymbol{\phi_q} gh^*) f_d  ~ds dt d\theta
	\\
	= &0& 
\end{array}
\end{equation}

On peut sortir $N_\phi(x)$ de la somme apportée par $U_{n-1}$. Ce qui permet de créer les termes :
\begin{equation}
\mathbf{K} = \int_\Omega \epsilon (N_\phi(x)) : H : \epsilon (N_\phi(x))~~dx
\end{equation}
\begin{equation}
\mathbf{C} = \int_\Omega N_\phi(x) \mu  N_\phi(x)~~dx
\end{equation}
\begin{equation}
\mathbf{M} = \int_\Omega N_\phi(x) \rho  N_\phi(x)~~dx
\end{equation}
\begin{equation}
\mathbf{f} = \int_\Omega N_\phi(x) f_v ~~dx 
	%+ \int_{\partial \Omega_U} U_d (H : \epsilon (N_\phi(x)) . n) 
	+ \int_{\partial \Omega_\mathbf{f}} N_\phi(x) f_d ~ds dt d\theta
\end{equation}

On a alors :
\begin{equation}
\begin{array}{r r l}
	&\int_T \! \int_\Theta \!\!\!&		
		gh\boldsymbol{\phi_q}^*
			[ ~\mathbf{K}~\boldsymbol{\phi_q} gh
			\\ && \phantom{gh\boldsymbol{\phi_q}^*[ }
			+ ~\mathbf{C}~ \boldsymbol{\phi_q}  \dot{g}h
			\\ && \phantom{gh\boldsymbol{\phi_q}^*[ }
			+ ~\mathbf{M}~ \boldsymbol{\phi_q} \ddot{g}h
			- N_\phi(x) f_v
			] ~dt d\theta
	\\
	+ &\int_T \! \int_\Theta \!\!\!&		
		gh\boldsymbol{\phi_q}^*
			[ ~\mathbf{K}~ \sum_{k=1}^{n-1} (\boldsymbol{\phi_q})_k       g_k  h_k
			\\ && \phantom{gh\boldsymbol{\phi_q}^*[}
			+ ~\mathbf{C}~ \sum_{k=1}^{n-1} (\boldsymbol{\phi_q})_k  \dot{g_k} h_k 
			\\ && \phantom{gh\boldsymbol{\phi_q}^*[}
			+ ~\mathbf{M}~ \sum_{k=1}^{n-1} (\boldsymbol{\phi_q})_k \ddot{g_k} h_k 
			] ~dt d\theta
	\\
	+ &\int_T \! \int_\Theta \!\!\!&		
		(g^*h + gh^*)\boldsymbol{\phi_q}
			[ ~\mathbf{K}~ \boldsymbol{\phi_q}gh
			\\ && \phantom{(g^*h + gh^*)\boldsymbol{\phi_q}[ }
			+ ~\mathbf{C}~ \boldsymbol{\phi_q} \dot{g}h
			\\ && \phantom{(g^*h + gh^*)\boldsymbol{\phi_q}[ }
			+ ~\mathbf{M}~ \boldsymbol{\phi_q} \ddot{g}h
			- N_\phi(x) f_v
			] ~dt d\theta
	\\
	+ &\int_T \! \int_\Theta \!\!\!&
		(g^*h + gh^*)\boldsymbol{\phi_q}
			[ ~\mathbf{K}~ \sum_{k=1}^{n-1} (\boldsymbol{\phi_q})_k      {g_k} h_k 
			\\ && \phantom{(g^*h + gh^*)\boldsymbol{\phi_q}[}
			+ ~\mathbf{C}~ \sum_{k=1}^{n-1} (\boldsymbol{\phi_q})_k  \dot{g_k} h_k 
			\\ && \phantom{(g^*h + gh^*)\boldsymbol{\phi_q}[}
			+ ~\mathbf{M}~ \sum_{k=1}^{n-1} (\boldsymbol{\phi_q})_k \ddot{g_k} h_k 
			] ~dt d\theta
	\\
	- &\int_{\partial \Omega_\mathbf{f}} \! \int_T \! \int_\Theta \!\!\!&
		N_\phi(x)(\boldsymbol{\phi_q}^*gh + \boldsymbol{\phi_q} g^*h + \boldsymbol{\phi_q} gh^*) f_d  ~ds dt d\theta
	\\
	= &0& 
\end{array}
\end{equation}

Puis avec $\mathbf{f}$ :
\begin{equation}
\begin{array}{r r l}
	&\int_T \! \int_\Theta \!\!\!&		
		gh\boldsymbol{\phi_q}^*[~\mathbf{K}~\boldsymbol{\phi_q} gh
			\\ && \phantom{gh\boldsymbol{\phi_q}^*[ }
			+ ~\mathbf{C}~ \boldsymbol{\phi_q}  \dot{g}h
			\\ && \phantom{gh\boldsymbol{\phi_q}^*[ }
			+ ~\mathbf{M}~ \boldsymbol{\phi_q} \ddot{g}h
			] ~dt d\theta
	\\
	+ &\int_T \! \int_\Theta \!\!\!&		
		gh\boldsymbol{\phi_q}^*
			[ ~\mathbf{K}~ \sum_{k=1}^{n-1} (\boldsymbol{\phi_q})_k       g_k  h_k
			\\ && \phantom{gh\boldsymbol{\phi_q}^*[}
			+ ~\mathbf{C}~ \sum_{k=1}^{n-1} (\boldsymbol{\phi_q})_k  \dot{g_k} h_k 
			\\ && \phantom{gh\boldsymbol{\phi_q}^*[}
			+ ~\mathbf{M}~ \sum_{k=1}^{n-1} (\boldsymbol{\phi_q})_k \ddot{g_k} h_k 
			] ~dt d\theta
	\\
	+ &\int_T \! \int_\Theta \!\!\!&		
		(g^*h + gh^*)\boldsymbol{\phi_q}[~\mathbf{K}~ \boldsymbol{\phi_q}gh
			\\ && \phantom{(g^*h + gh^*)\boldsymbol{\phi_q}[ }
			+ ~\mathbf{C}~ \boldsymbol{\phi_q} \dot{g}h
			\\ && \phantom{(g^*h + gh^*)\boldsymbol{\phi_q}[ }
			+ ~\mathbf{M}~ \boldsymbol{\phi_q} \ddot{g}h
			] ~dt d\theta
	\\
	+ &\int_T \! \int_\Theta \!\!\!&
		(g^*h + gh^*)\boldsymbol{\phi_q}
			[ ~\mathbf{K}~ \sum_{k=1}^{n-1} (\boldsymbol{\phi_q})_k       g_k  h_k 
			\\ && \phantom{(g^*h + gh^*)\boldsymbol{\phi_q}[}
			+ ~\mathbf{C}~ \sum_{k=1}^{n-1} (\boldsymbol{\phi_q})_k  \dot{g_k} h_k 
			\\ && \phantom{(g^*h + gh^*)\boldsymbol{\phi_q}[}
			+ ~\mathbf{M}~ \sum_{k=1}^{n-1} (\boldsymbol{\phi_q})_k \ddot{g_k} h_k 
			] ~dt d\theta
	\\
	- &\int_T \! \int_\Theta \!\!\!&
		~\mathbf{f}~ (\boldsymbol{\phi_q}^*gh + \boldsymbol{\phi_q} g^*h + \boldsymbol{\phi_q} gh^*) dt d\theta
	\\
	= &0& 
\end{array}
\end{equation}

On peut alors réarranger les lignes :
\begin{equation}
\begin{array}{r r l}
	&\int_T \! \int_\Theta \!\!\!&		
		gh\boldsymbol{\phi_q}^*[~\mathbf{K}~\boldsymbol{\phi_q} gh + ~\mathbf{C}~ \boldsymbol{\phi_q}  \dot{g}h 
				+ ~\mathbf{M}~ \boldsymbol{\phi_q} \ddot{g}h
				] ~dt d\theta
	\\
	+ &\int_T \! \int_\Theta \!\!\!&		
		gh\boldsymbol{\phi_q}^*[  ~\mathbf{K}~ \sum_{k=1}^{n-1} (\boldsymbol{\phi_q})_k       g_k  h_k 
				+ ~\mathbf{C}~ \sum_{k=1}^{n-1} (\boldsymbol{\phi_q})_k  \dot{g_k} h_k 
				+ ~\mathbf{M}~ \sum_{k=1}^{n-1} (\boldsymbol{\phi_q})_k \ddot{g_k} h_k 
				] ~dt d\theta
	\\
	+ &\int_T \! \int_\Theta \!\!\!&		
		(g^*h + gh^*)\boldsymbol{\phi_q}[~\mathbf{K}~ \boldsymbol{\phi_q}gh
						+ ~\mathbf{C}~ \boldsymbol{\phi_q} \dot{g}h 
						+ ~\mathbf{M}~ \boldsymbol{\phi_q} \ddot{g}h
						] ~dt d\theta
	\\
	+ &\int_T \! \int_\Theta \!\!\!&
		(g^*h + gh^*)\boldsymbol{\phi_q}[ ~\mathbf{K}~ \sum_{k=1}^{n-1} (\boldsymbol{\phi_q})_k       g_k  h_k 
						+ ~\mathbf{C}~ \sum_{k=1}^{n-1} (\boldsymbol{\phi_q})_k  \dot{g_k} h_k 
						+ ~\mathbf{M}~ \sum_{k=1}^{n-1} (\boldsymbol{\phi_q})_k \ddot{g_k} h_k 
						] ~dt d\theta
	\\
	- &\int_T \! \int_\Theta \!\!\!&
		~\mathbf{f}~ (\boldsymbol{\phi_q}^*gh + \boldsymbol{\phi_q} g^*h + \boldsymbol{\phi_q} gh^*) ~dt d\theta
	\\
	= &0& 
\end{array}
\end{equation}

Et en factorisant :
\begin{equation}
\!\!\!\!\!\!\!\!\!\!\!\!\!\!\!\!
\begin{array}{r r l}
	&\int_T \! \int_\Theta \!\!\!&		
		(\boldsymbol{\phi_q}^*gh + \boldsymbol{\phi_q}g^*h + \boldsymbol{\phi_q}gh^*)[~\mathbf{K}~ \boldsymbol{\phi_q}gh
						+ ~\mathbf{C}~ \boldsymbol{\phi_q} \dot{g}h 
						+ ~\mathbf{M}~ \boldsymbol{\phi_q} \ddot{g}h
						] ~dt d\theta
	\\
	+ &\int_T \! \int_\Theta \!\!\!&
		(\boldsymbol{\phi_q}^*gh + \boldsymbol{\phi_q}g^*h + \boldsymbol{\phi_q}gh^*)
			[ ~\mathbf{K}~ \sum_{k=1}^{n-1} (\boldsymbol{\phi_q})_k       g_k  h_k
			+ ~\mathbf{C}~ \sum_{k=1}^{n-1} (\boldsymbol{\phi_q})_k  \dot{g_k} h_k 
			+ ~\mathbf{M}~ \sum_{k=1}^{n-1} (\boldsymbol{\phi_q})_k \ddot{g_k} h_k 
			] ~dt d\theta
	\\
	- &\int_T \! \int_\Theta \!\!\!&
		(\boldsymbol{\phi_q}^*gh + \boldsymbol{\phi_q} g^*h + \boldsymbol{\phi_q} gh^*) ~\mathbf{f}~ ~dt d\theta
	\\
	= &0& 
\end{array}
\end{equation}

Puis en regroupant les intégrales :
\begin{equation}
\begin{array}{r l}
	\int_T \! \int_\Theta \!\!\!&		
		(\boldsymbol{\phi_q}^*gh + \boldsymbol{\phi_q}g^*h + \boldsymbol{\phi_q}gh^*)[~\mathbf{K}~ \boldsymbol{\phi_q}gh
						+ ~\mathbf{C}~ \boldsymbol{\phi_q} \dot{g}h 
						+ ~\mathbf{M}~ \boldsymbol{\phi_q} \ddot{g}h
	\\
	  &
		\phantom{(\boldsymbol{\phi_q}^*gh + \boldsymbol{\phi_q}g^*h + \boldsymbol{\phi_q}gh^*)
			}+ \mathbf{K}~ \sum_{k=1}^{n-1} (\boldsymbol{\phi_q})_k       g_k  h_k 
			+ ~\mathbf{C}~ \sum_{k=1}^{n-1} (\boldsymbol{\phi_q})_k  \dot{g_k} h_k 
			+ ~\mathbf{M}~ \sum_{k=1}^{n-1} (\boldsymbol{\phi_q})_k \ddot{g_k} h_k
	\\
	  &
		\phantom{(\boldsymbol{\phi_q}^*gh + \boldsymbol{\phi_q} g^*h + \boldsymbol{\phi_q} gh^*)} -\mathbf{f}~] ~dt d\theta
	\\
	= &0
\end{array}
\end{equation}
%\end{document}